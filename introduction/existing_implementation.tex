\section{Existing Implementation}
In order to study the effects of impact ionisation in solar cells by means of small Hubbard clusters a program was developed. The original numerical implementation was by Michael Innerberger \cite{innerberger} and extensions were carried out by Paul Worm \cite{worm_bachelor, worm_project} and Paul Prauhart \cite{prauhart}.
\medskip

In this section we will only give a brief overview of the main functionality of the code and refer the reader back to the previously mentioned three papers for more details.

\subsection{Hubbard Model} \label{sec:hubbard_model}
To study the effect of impact ionisation in strongly correlated electron system the Hubbard model was used. It describes system of atoms as a lattice of $N_s\in \mathbb{N}$ sites, each of which can represent the location of at most one spin-up electron and one spin-down electron under the tight-binding approximation. The wave functions that describe such localised electrons are called Wannier-Functions.
\medskip

Using the second quantisation formalism of quantum mechanics, states of such many-body systems can be described purely by the occupation numbers of each site $n_{i\sigma} = \{0,1\}$. This allows us to write state vectors as $\ket{\psi}= \ket{n_{1\uparrow}n_{1\downarrow} n_{2\uparrow}n_{2\downarrow}\dots n_{N_{s}\uparrow}n_{N_{s}\downarrow}}$. The Hamiltonian can then be expressed using the following two terms
\begin{equation}
    \hat{H}_{\text{Hubbard}} = U \sum_i \hat{n}_{i\uparrow} \hat{n}_{i\downarrow} + \sum_{ij\sigma} v_{ji} \hat{c}^\dagger_{i\sigma} \hat{c}_{j\sigma}\label{eq:hubbard_hamiltonian}
\end{equation}
The first corresponding to the repulsive Coulomb interaction $U$ between two electrons with opposite spins located on the same site. The second represents the energy change in the system when an electron hops form site $j\to i$ with hopping amplitude $v_{ji}$. Here $\hat{c}_{i\sigma}^\dagger, \hat{c}_{j\sigma}$ are the fermionic creation and annihilation operators.
\medskip

One can interpret the first sum as representing the potential (or interaction) energy in the system where the second sum for the hoppings represents the systems kinetic energy, since it is related to movement of electrons. This distinction becomes important in section \ref{sec:non_local_coulomb} where we introduce non-local coulomb interactions.



\subsection{Interaction with Photons}
The systems interaction with photons is modelled via a classical electric field pulse
\begin{equation}
    \Vec{E}(t) = \Vec{E}_0 \sin(\omega(t-t_p))\operatorname{e}^{-\frac{(t-t_p)^2}{2\sigma^2}} \label{eq:e_field}
\end{equation}
with frequency $\omega$, width $\sigma$ and peak time $t_p$. This can be integrated into our model by introducing a time dependent phase factor onto the hopping amplitude in a method called Peierl's substitution \cite{peierl}.
\begin{equation}
    v_{ij} \to v_{ij}(t) = v_{ij}\exp\left(i\frac{e}{\hbar} \int_{\Vec{R}_i}^{\Vec{R}_j} \Vec{A}(\Vec{r},t) d\Vec{r}\right)\label{eq:hopping}
\end{equation}
Here $\Vec{A}$ is the electromagnetic vector potential which, in a gauge where the scalar potential vanishes, can be expressed via $\Vec{E}(t) = -\partial_t \Vec{A}(t)$. Using an approximation for the integral in (\ref{eq:hopping}) and the equation (\ref{eq:e_field}) one arrives at
\begin{equation}
    v_{ij} \approx v_{ij} \exp\left(ia[\cos(\omega (t-t_p))-b] e^{-\frac{(t-t_p)^2}{2\sigma^2}}\right)
\end{equation}
where $a$ and $b$ are tunable parameters.


\subsection{Time-Evolution}

Our main interest is to investigate impact ionisation in the system, after being exposed to the electric field pulse. Prior to this work it was achieved by looking at the double occupation observable $\braket{\hat{d}(t)} = \braket{\hat{n}_{i\uparrow}(t) \hat{n}_{i\downarrow}(t)}$ since the rise of its mean value after initial excitation was an indicator for impact ionisation. In order to compute this quantity, we need the ability to time-evolve any initial state $\ket{\psi(t=0)} = \ket{\psi_0}$ of the system. Using the time dependent Schrödinger equation
\begin{equation}
    i\hbar \frac{\partial}{\partial t}\ket{\psi(t)} = \hat{H}(t)\ket{\psi(t)}\label{eq:time_evolve}
\end{equation}
one finds that the time evolution can be computed by
\begin{equation}
    \ket{\psi (t)} = \mathcal{T} \exp\left(-\frac{i}{\hbar}\int_0^t H(t') dt'\right) \ket{\psi_0}
\end{equation}
where $\mathcal{T}$ is the time-ordering operator. By numerically deviding $t$ into $m$ small time steps $\tau$ it is justified to use Magnus-expansion \cite{magnus} of order zero and thus neglect the time-ordering operator. Assuming the time steps are small enough we can approximate the integral in (\ref{eq:time_evolve}) over one time step using the midpoint rule. This leads to the following recursive formula
\begin{equation}
    \ket{\psi(m\tau + \tau)} = \exp\left(-\frac{i}{\hbar} H\left(m\tau + \frac{\tau}{2}\right)\right)\ket{\psi(m\tau)}
\end{equation}
Which can be computed using the Krylov matrix exponential method \cite{innerberger}.


\subsection{Finding the Initial State}
For impact ionisation we mainly concern ourselves with half-filled systems, where the number of spin-up and spin-down electrons are the same. This is an invariant property of the systems and does not change over time.
\medskip

It is assumed that before the electric field pulse, the system is in thermal equilibrium and occupies the state with the lowest energy (the ground state). To find this state numerically a variant of the power iteration method was implemented \cite{innerberger}. From a randomly initialised starting state, it iterativley computes the eigenenergy with the largest absolute value and its corresponding eigenstate, thus recursively converging on to the ground state.


\subsection{Memory Management}
The predominant limitation of this implementation is the memory it requires. For a system with $N_s$ sites and a fixed number of electrons the dimension of its Hilbert space , i.e. the number of linearly independent states, is given by 
\begin{equation}
    \dim \left[\mathcal{H}^{n_\uparrow}_{n_\downarrow} (N_s)\right] = \begin{pmatrix}N_s \\ n_\uparrow\end{pmatrix} \begin{pmatrix}N_s \\ n_\downarrow\end{pmatrix}
\end{equation}
which for a half-filled system with $14$ sites is about $\SI{12e6}{}$. A Hamiltonian matrix of that size, with elements of the data type double, would take over $1000$ Terabytes of memory. However due to the Hamiltonian being made up of mostly zeros, it can be stored and manipulated in a highly compressed sparse matrix format \cite{innerberger}, substantially reducing the memory needs.
\medskip

The states of the system are stored as integers, whose binary representations correspond to the occupation configuration of the sites. Actions on these states like creation, annihilation and hoppings have been implemented as bitwise operations.
\medskip
 
With all these data saving measures in place, it was possible to compute the time evolution of 2D square lattices and chains of up to 14 sites, above which memory becomes the limiting factor once again.


\subsection{Spectral Functions}\label{sec:spectral_functions}

One way of obtaining information about the states of a system is via spectral functions $A(\omega, t)$. They provide insight about which states the electrons in the system can occupy, and thus also tell us about the energies required for electrons to transitions to occur. The existing codebase offers two methods of computing spectral functions \cite{spectral_function}

\subsubsection{Lehmann Spectra}
Implemented by M. Innerberger in \cite{innerberger}, the Lehmann representation of the spectral function gives the equilibrium, i.e. time-independent, spectrum of the system. 
\begin{equation}
    A(E)=\sum_{i, \sigma} \sum_{|\phi\rangle}\left(\left|\left\langle\phi\left|\hat{c}_{i \sigma}^{\dagger}\right| \psi_{0}\right\rangle\right|^{2} \delta\left(E-E_{|\phi\rangle}+E_{0}\right)+\left|\left\langle\phi\left|\hat{c}_{i \sigma}\right| \psi_{0}\right\rangle\right|^{2} \delta\left(E+E_{|\phi\rangle}-E_{0}\right)\right)
\end{equation}
Where $\{ \ket{\phi}\}$ is an eigenbasis of $\mathcal{H}(N_s)$ with respective energy $E_{\ket{\phi}}$, and $\ket{\psi_0}$ is the ground state of $\hat{H}$ with energy $E_0$.

\bigskip
It is simple and quick to generate with the existing code and produces numerically elegant results. However, because of it being quite memory intensive, it can not handle systems with large numbers of sites. This together with the fact that we can not investigate the time dependence of the spectrum, prompted the implementation of a way to calculate the non-equilibrium spectral function.

\subsubsection{Fourier Transforms of Non-Equilibrium Green's Functions}

Originally implemented by P. Worm \cite{worm_bachelor}, with efficiency improvements by P. Prauhart \cite{prauhart}, the one-particle non-equilibrium greens function $G_{ij\sigma} (t,t')$ can be used to compute the non-equilibrium spectral function $A_{ij\sigma}(\omega, t)$, via forward Fourier transformation.

\begin{equation}
    G_{i j \sigma}^{<}\left(t, t^{\prime}\right)=\mathrm{i}\left\langle\psi\left(t^{\prime}\right)\left|\hat{c}_{j \sigma}^{\dagger} \mathcal{T} e^{-\mathrm{i} \int_{t}^{t^{\prime}} H(\tau) \mathrm{d} \tau} \hat{c}_{i \sigma}\right| \psi(t)\right\rangle\label{eq:greens_function_lesser}
\end{equation}

\begin{equation}
    G_{i j \sigma}^{>}\left(t, t^{\prime}\right)=\mathrm{i}\left\langle\psi\left(t^{\prime}\right)\left|\hat{c}_{j \sigma} \mathcal{T} e^{-\mathrm{i} \int_{t}^{t^{\prime}} H(\tau) \mathrm{d} \tau} \hat{c}_{i \sigma}^\dagger\right| \psi(t)\right\rangle\label{eq:greens_function_greater}
\end{equation}

\begin{equation}
    A_{ij\sigma}^\gtlt = \frac{1}{\pi} \operatorname{Im}\int_0^\infty \operatorname{e}^{i\omega t_{rel}} G_{ij\sigma}^\gtlt (t,t+t_{rel}) dt_{rel}\label{eq:non_equilibirum_spectral_function}
\end{equation}

A simple interpretation of the definition in \ref{eq:greens_function_lesser} is that an electron with spin $\sigma$ is removed from site $i$ of the system at time $t$, the system is time evolved to $t'$ and then the electron is added back in at site $j$. $G_{ij\sigma}^{<}$ is then proportional to the probability that this system is in the same state as a system at time $t'$ where the electron was never removed. The exact implementation is elaborated on in \cite{worm_bachelor}. 

\subsubsection{Obtaining spectral functions using the existing code}

In the current state of the code the Lehmann spectrum can be generated automatically and is output as a .dat file containing columns of values of $A(\omega)$ for each site.

\bigskip
The Green's function is also output as a .dat file with columns containing complex number values of $G(t,t'=t_0)$ for each site. In order to obtain the spectral function from this, one needs to perform the Fourier transform of this data manually, using (\ref{eq:non_equilibirum_spectral_function}), as there is currently no existing function in the codebase that handles this. In order to obtain more usable, less noisy results one should, prior to the Fourier transform, multiply $G$ with a dampening factor $e^{-\epsilon (t-t_0)}$, which effectively acts as a Lorentzian broadening in frequency space \cite{spectral_function}. To achieve a higher frequency/energy resolution one can additionally zero-pad the domain before performing the Fourier transform. 