\section{Quantum Dot}

Quantum dots (QDs) are nanoscopic semiconducting particles that display interesting quantum mechanical effects. Due to their small size, the energy levels of the electrons within them become quantized, much like in atoms. Unlike atoms however, the size of QDs can be chosen arbitrarily, making their absorption spectra highly tunable.  Together with cost-effective manufacturing processes this makes QDs an enticing candidate for new photovoltaic technologies.
\medskip

Their high efficiency in energy conversion has been shown experimentally, and impact ionisation has been proposed as an explanation \cite{impact_io_in_qd}. In this work we first implement a quantum dot into our model, and investigate whether it can facilitate impact ionisation. Subsequently we couple the quantum dot to a larger molecule, benzene in this case, and investigate the effect of this on impact ionisation. This is motivated by the experimental work done in \cite{qd_motivation} where impact ionisation was seen in such a system.

\subsection{Geometry}
 
 In implementing quantum dots, one of two choices had to be made. Either we implement a new type of site, with multiple energy levels, or we model the quantum dot as a particular arrangement of sites, which we engineer to have the desired effective energy levels. The latter option requires less alterations to the existing code and thus was chosen.
 
 \medskip
 
 For all the computations done in this work the QD is modelled as a $2\times 2$ lattice of sites, with the following horizontal, vertical and diagonal hopping amplitudes: $v_h = v_v = 1$, $v_d = 0$.
 
 \medskip
 
 In order to be consistent with the model used for the Benzene ring, the nonlocal Coulomb interactions for the QD are also calculated using the PPP model as described in section \ref{subsec:ppp}. The distances used in the Ohno function, shown in Fig.
 \ref{fig:qd_distances}, were calculated by choosing the lattice constant to be equal to the $C-C$ bond length $d =\SI{1.4}{\angstrom}$. This does not necessarily have a physical basis, but for now it provides sufficient flexibility to engineer the desired spectrum by varying the on-site Coulomb interaction $U$, which is shown in section \ref{subsec:spectrum_engineering}.
 
\begin{figure}[!hbt]
    \begin{minipage}[b]{.48\textwidth}
        \centering
    \begin{tikzpicture}[
                    QD/.style={draw,thick,circle, fill=black!40, inner sep=3.5},
                    QDlabel/.style={label={above left:{\color{gray}#1}}},
                    QDlabelRight/.style={label={above right:{\color{gray}#1}}},
                    arrow/.style={<->,>=Latex, thick, yellow!40!green},
                    arrow_label/.style={midway, circle, fill=white, inner sep=0}
                    ]
    \newdimen \qdd
    \qdd = 2cm
    %QD nodes    
    \node[QD, label={above left:{\color{gray}0}}]  at (0,\qdd)       (QD0){};
    \node[QD, label={above right:{\color{gray}1}}]  at (\qdd,\qdd)    (QD1){};
    \node[QD, label={below left:{\color{gray}2}}]   at (0,0)          (QD2){};
    \node[QD, label={below right:{\color{gray}3}}]  at (\qdd,0)       (QD3){};
    
    \draw[thick] (QD2.east) -- (QD3.west);
    \draw[thick] (QD0.east) -- (QD1.west);
    \draw[thick] (QD0.south) -- (QD2.north);
    \draw[thick] (QD1.south) -- (QD3.north);
    
    \draw[arrow] (QD2) -- (QD0) node[arrow_label] {$d$};
    \draw[arrow] (QD2) -- (QD1) node[arrow_label] {$\sqrt{2}d$};   
    
    \end{tikzpicture}
    
    \caption{Distances used in Ohno function (\ref{eq:ohno_interpolation}) to calculate the coulomb potential $U_{ij}$. With $d = \SI{1.4}{\angstrom}$\newline}
    \label{fig:qd_distances}
    \end{minipage}
    \hfill
    \begin{minipage}[b]{.48\textwidth}
        \centering
    \begin{tikzpicture}[
                    QD/.style={draw,thick,circle, fill=black!40, inner sep=3.5},
                    QDlabel/.style={label={above left:{\color{gray}#1}}},
                    QDlabelRight/.style={label={above right:{\color{gray}#1}}},
                    arrow/.style={->, thick, yellow!40!green},
                    arrow_label/.style={midway, circle, fill=white, inner sep=0},
                    darrow_label/.style={pos=0.45, circle, fill=white, inner sep=1pt},
                    vhop/.style={->,>=Latex, thick, red, shorten <=0.2cm, shorten >=0.2cm},
                    hhop/.style={->,>=Latex, thick, cyan!70!red, shorten <=0.2cm, shorten >=0.2cm},
                    dhop/.style={->,>=Latex, thick, orange!60!yellow, shorten <=0.2cm, shorten >=0.2cm},
                    dhop2/.style={thick, orange!60!yellow, shorten <=0.2cm, shorten >=0.2cm}
                    ]
    \newdimen \qdd
    \qdd = 2cm
    %QD nodes    
    \node[QD, label={above left:{\color{gray}0}}]  at (0,\qdd)       (QD0){};
    \node[QD, label={above right:{\color{gray}1}}]  at (\qdd,\qdd)    (QD1){};
    \node[QD, label={below left:{\color{gray}2}}]   at (0,0)          (QD2){};
    \node[QD, label={below right:{\color{gray}3}}]  at (\qdd,0)       (QD3){};
    
    % vertical and horizonatal hoppings
    \draw[vhop] (QD1.south) --  (QD3.north) node[arrow_label] {$v_v$};
    \draw[vhop] (QD0.south) --  (QD2.north);
    \draw[hhop] (QD0.east)  --  (QD1.west)  node[arrow_label] {$v_h$};
    \draw[hhop] (QD2.east)  --  (QD3.west);
    
    % diagonal hoppings
    \draw[dhop2] (QD1.south west) to[out=215, in=55]                             (QD2.north east);
    \draw[dhop] (QD0.south east) to[out=-35, in=125] node[darrow_label] {$v_d$} (QD3.north west);
    
    
    \end{tikzpicture}
    \caption{Possible hoppings between QD sites. Arrow direction indicates the direction in which the sign of the imaginary part of the hopping amplitude is positive.}
    \label{fig:qd_hoppings}
    \end{minipage}
\end{figure}

 \subsection{Light pulse direction and hopping time dependence}
For every geometry, it is necessary to define a function that explicitly determines whether or not the hopping amplitude $v_{ij}$ between two any two sites $i$ and $j$ is time dependent, i.e. is modified by the light pulse.
\medskip

This depends on the direction of the light pulse, which for our implementation is always along the diagonal coming from the top left. Any hoppings perpendicular to the pulse direction will remain constant over time and are thus not time dependent.
\medskip

The sign of the imaginary parts of the hopping amplitudes is positive for hoppings parallel to the pulse direction and negative for hoppings perpendicular to the pulse direction. By labelling the sites in lexicographical order as shown in Fig. \ref{fig:qd_hoppings} we ensure that all hoppings from sites $i\to j$ have the same sign as $(j-i)$. The positive direction of the imaginary part of the hopping amplitudes are indicated in Fig. \ref{fig:qd_hoppings} by arrowheads. The diagonal hopping between sites 1 and 2 has no direction shown because in our case it is perpendicular to the light pulse and will have no effect anyway.

