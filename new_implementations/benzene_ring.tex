\section{Benzene Ring}

%%%%%%%%%%%%%%%%%%%%%%%%%%%%%%%%%%%%%%%%%%%%%%%%%%%%%%%%%%%%%%%%%%%%%%%%%%%%%%%%%%





\subsection{Hexagonal Geometry}

We model the Benzene ring as a 6-site chain with periodic boundary conditions. This has the benefit that it is simple to implement in the existing framework, as a 6x1 lattice with an extra hopping between sites $0$ and $5$. 


\medskip

The energy of the non-local coulomb interactions $U_{ij}$ are calculated using the Ohno interpolation (\ref{eq:ohno_interpolation}). The distances $|r_{ij}|$ are calculated using the C-C bond length in the benzene molecule, $d=\SI{1.40}{\angstrom}$, and its hexagonal geometry, shown in Fig. \ref{fig:benzene_distances}

\begin{figure}[!hbt]
    \centering
    \begin{tikzpicture}[Benzene/.style={draw,thick, circle, radius = .1em,                                      fill=red!80!yellow},
                    arrow/.style={->, >=Latex,thick, yellow!40!green},
                    arrow_label/.style={midway, circle, fill=white, inner sep=0}
                    ]
        \newdimen\R
        \R=1.5cm
        \draw[thick] (0:\R)
        \foreach \x in {60,120,...,360} {  -- (\x:\R) }
        -- cycle (360:\R) node[Benzene] (Bc3){}
        -- cycle (300:\R) node[Benzene] (Bc4){}
        -- cycle (240:\R) node[Benzene] (Bc5){}
        -- cycle (180:\R) node[Benzene] (Bc0){}
        -- cycle (120:\R) node[Benzene] (Bc1){}
        -- cycle  (60:\R) node[Benzene] (Bc2){};
        
        \draw[arrow] (Bc0) -- (Bc1) node[arrow_label] {$d$};
        \draw[arrow] (Bc0) -- (Bc2) node[arrow_label] {$\sqrt{3}d$};
        \draw[arrow] (Bc0) -- (Bc3) node[arrow_label] {$2d$};
    \end{tikzpicture}
    \caption{Interatomic distances in Benzene ($d=\SI{1.40}{\angstrom})$}
    \label{fig:benzene_distances}
\end{figure}

\subsection{Hopping amplitudes}
One problem with modelling the benzene ring as a chain is that we lose the orientation of the lines connecting two sites relative to the light pulse direction.
This leads to an inaccuracy in the angle dependent time dependence of the hoppings between the sites. However, by comparing the results with those of another group, where the orientation was taken into account, we saw only negligible differences in the results and thus choose to move forward with this model.
%\medskip
%lexicographical indexing no longer produces proper hopping signs... but as mentioned we ignore that for now
    %%%%%%%%%%%%%%%%%
    % PULSE ON LINE %
    %%%%%%%%%%%%%%%%%
    
\begin{figure}[!hbt]
    \centering
    \begin{tikzpicture}[decoration=snake,
                    Benzene/.style={draw,thick, circle, radius = .1em, fill=red!80!yellow},
                    squiggly/.style={->, decorate, thick, yellow!70!red},
                    doublearrow/.style={->, double, line width=1pt, -Implies, double distance=3pt, shorten >= 2cm, shorten <=2cm}
                    ]
                    
    
    % RING %
    \newdimen\R
    \R=1.5cm
    \draw[thick] (0:\R)
    \foreach \x in {60,120,...,360} {  -- (\x:\R) }
    -- cycle (360:\R) node[Benzene, label=3] (Bc3){}
    -- cycle (300:\R) node[Benzene, label=4] (Bc4){}
    -- cycle (240:\R) node[Benzene, label=5] (Bc5){}
    -- cycle (180:\R) node[Benzene, label=0] (Bc0){}
    -- cycle (120:\R) node[Benzene, label=1] (Bc1){}
    -- cycle  (60:\R) node[Benzene, label=2] (Bc2){};
    
    \foreach \x/\y in {0/1,1/2,2/3,3/4,4/5,5/0}{
    \draw[squiggly] ($(Bc\x)!0.5!(Bc\y) + (-1,1)$) -- ($(Bc\x)!0.5!(Bc\y)$);
    }
    
    
    % CHAIN DOWN%
    \foreach \x in {0,1,2,3,4,5}
    \node[Benzene, label=below:\x] at ($(1.3*\x,0) + (7,-1.3)$)  (B\x){};
    
    \foreach \x/\y in {0/1, 1/2, 2/3, 3/4, 4/5}
    \draw[squiggly] ($(B\x)!0.5!(B\y) + (-1, 1)$) -- ($(B\x)!0.5!(B\y)$);
    
    \foreach \x/\y in {0/1,1/2,2/3,3/4,4/5}
    \draw[thick] (B\x) -- (B\y);
    
    \draw[dashed] ($(B0) - (0.7,0)$) -- (B0);
    \draw[dashed] (B5) -- ($(B5) + (0.7,0)$);
    
    % CHAIN UP %
    \foreach \x in {0,1,2,3,4,5}
    \node[Benzene, label=below:\x] at ($(1.3*\x,0) + (7,1.3)$)  (Bu\x){};
    
    \foreach \x/\y/\angle in {0/1/80, 1/2/135, 2/3/190, 3/4/260, 4/5/340}
    \draw[squiggly] ($(Bu\x)!0.5!(Bu\y) + (\angle:1.4)$) -- ($(Bu\x)!0.5!(Bu\y)$);

    \foreach \x/\y in {0/1,1/2,2/3,3/4,4/5}
    \draw[thick] (Bu\x) -- (Bu\y);

    \draw[dashed] ($(Bu0) - (0.7,0)$) -- (Bu0);
    \draw[dashed] (Bu5) -- ($(Bu5) + (0.7,0)$);
    

    % DOUBLE ARROWS %
    \draw[doublearrow] (Bc3) -- (Bu0) node[pos=0.5,sloped,rotate=150]{\color{red}\Huge$|$}node[pos=0.5,rotate=40]{\color{red}\Huge$|$};
    \draw[doublearrow] (Bc3) -- (B0);
    \end{tikzpicture}
    \caption{Caption}
\end{figure}