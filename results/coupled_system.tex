\section{Coupled System}

\subsection{Testing}
Before we investigate the effects of increasing the coupling strength between the two systems, we verify that if the hopping amplitude $v_c = 0$, the spectra on the individual sites remain unchanged. For the coupled system, calculating the Lehmann spectrum is no longer an option as memory becomes a limiting factor, instead we resort to using the Green's function method.

\medskip
As can be seen when comparing Fig. \ref{fig:coupled_vc_0} with the spectra in Fig. \ref{fig:spectrum_engineering} and Fig. \ref{fig:isolated_benzene}, this is indeed the case. The slight variations being due to a difference in parameters used to perform the Fourier transformation.

\begin{figure}[!hbt]
    \centering
    \includegraphics[width=0.49\textwidth]{graph/coupled_QD_vc_0.pdf}
    \includegraphics[width=0.49\textwidth]{graph/coupled_benzene_vc_0.pdf}sh
    \caption{Spectral function of all the sites of the coupled system with the coupling parameter turned off, $v_c = 0$. These spectra should be identical to those in Fig. \ref{fig:spectrum_engineering} and Fig. \ref{fig:isolated_benzene}}
    \label{fig:coupled_vc_0}
\end{figure}

\subsection{Dependence of the spectrum on the coupling strength}

\begin{figure}[!hbt]
    \centering
    \includegraphics[width=0.49\textwidth]{graph/spectrum_vc_sweep/QD_All_020.pdf}
    \includegraphics[width=0.49\textwidth]{graph/spectrum_vc_sweep/QD_All_030.pdf}
    \includegraphics[width=0.49\textwidth]{graph/spectrum_vc_sweep/QD_All_070.pdf}
    \includegraphics[width=0.49\textwidth]{graph/spectrum_vc_sweep/QD_All_100.pdf}
    \caption{}
    \label{fig:spectrum_vc_sweep}
\end{figure}